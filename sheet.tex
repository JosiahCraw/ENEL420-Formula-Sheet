\documentclass[12pt]{article}
\usepackage{graphicx}
\usepackage[section]{placeins}
\graphicspath{ {./img/} }
\usepackage[a4paper]{geometry}
\usepackage{hyperref}
\usepackage[utf8]{inputenc}
\usepackage{tabto}
\usepackage[english]{babel}
\usepackage{fancyhdr}
\usepackage{chemformula}
\usepackage{xcolor}
\usepackage{amsmath}
\usepackage{parskip}
\usepackage{float}
\usepackage{tabto}
\usepackage{subcaption}
\usepackage{tikz}
\usepackage[section]{placeins}
\usepackage{booktabs}
\usepackage{adjustbox}
\usepackage{array}
\usepackage{gensymb}
\usepackage{array}
\usetikzlibrary{shapes, arrows}

% Document Parameters
\newcommand\reportTitle{ENEL420 Formula Sheet}
\newcommand\subTitle{The bitchiest of all the exams}

\newcommand\todo[1]{\textcolor{red}{#1}}


\author{{\LARGE Josiah Craw}\vspace{10pt}\\35046080\\jcr124@uclive.ac.nz\\}
\title{\rule{\textwidth}{0.8pt} \\ {\huge \textbf{\reportTitle}}\\{\large \subTitle} \rule{\textwidth}{0.8pt}}

\begin{document}
% Titlepage
\maketitle
\newpage

\section{Interpolation}
Interpolation, convolution with a kernel function:
\begin{equation}
f(k_0) = \sum_k f(k) \frac{K(k_0-k)}{\textrm{convolution kernel}} 
\end{equation}
Where $f(k_0)$ is the interpolated signal and $f(k)$ is the data

\subsection{Kernels}

Nearest Neighbour:
\begin{equation}
K(x)=\textrm{rect}(x) =
\begin{cases}
    1 \quad |x| \leq \frac{1}{2}\\
    0 \quad \textrm{otherwise}
\end{cases}
\end{equation}

Linear:
\begin{equation}
    K(x)=\textrm{tri}(x) =
    \begin{cases}
        1 - |x| \quad |x| < 1\\
        0 \quad \textrm{otherwise}
    \end{cases}
\end{equation}

Sinc:
\begin{equation}
    K(x) = \textrm{sinc}(x) = \frac{\textrm{sin}(\pi x)}{\pi x}
\end{equation}

\section{Aliasing}
General Case to avoid aliasing:
\begin{equation}
    \frac{2f_u}{n} < f_s < \frac{2f_l}{n-1}
\end{equation}

Where,
\begin{equation}
    1 \leq n \leq I_{MLT}\left(\frac{f_u}{B}\right)
\end{equation}

$I_{MLT}$ is Maximum integer less than

\section{Z-Transform}
\begin{equation}
    F(z)=\sum_{n=0}^{\infty}f(n)z^{-n}
\end{equation}

Or,
\begin{equation}
    F(s)=\sum_{n=0}^{\infty}f(nT)e^{-nTs}
\end{equation}

\section{Quantisation Noise}

RMS quantisation noise:
\begin{equation}
    \sigma = \frac{\Delta}{\sqrt{12}} = \frac{\Delta}{2\sqrt{3}}
\end{equation}

Noise Power:
\begin{equation}
    \textrm{NP} \propto \Delta^2
\end{equation}

Therefore is noise power is decreased by $\frac{1}{2}$,
\begin{equation}
    10\textrm{log}_{10}\left(\frac{1}{2}^2\right) = 6\textrm{dB}
\end{equation}

\section{Filters}

Direct realisation:
\begin{equation}
    H(z) = \frac{\sum_{k=0}^K b_k z^{-k}}{1+\sum_{m=1}^M a_m z^{-m}}
\end{equation}

Linear Phase, if a filter has linear phase it can be expressed in the form:
\begin{equation}
    H(\omega) = |H(\omega)|e^{j\theta(\omega)}
\end{equation}

Where,
\begin{equation}
    \theta(\omega) = -(\alpha\omega+\beta)
\end{equation}

Windowing the impulse response is defined by the inverse Fourier Transform,
\begin{equation}
    h_D(n) = \frac{1}{2\pi} \int_{-\pi}^\pi H_D(\omega)e^{j\omega n} \quad d\omega 
\end{equation} 

non-recursive FIR Filters,
\begin{equation}
    y(n) = b_0 x(n) + b_1 x(n-1)
\end{equation}

Recursive,
\begin{equation}
    y(n) = a y(n-1) + bx(n)
\end{equation}

So,
\begin{equation}
    H(z) = \sum_{n=0}^{N-1} \frac{1}{N} \sum_{k=0}^{N-1} H(k) e^{\frac{j2\pi kn}{N}}z^{-n} 
\end{equation}

\section{Detection Theory}

Likelihood Ratio,
\begin{equation}
    \frac{p(\textbf{x}|H_0)}{p(\textbf{x}|H_1)}
\end{equation}

\subsection{Likelihood ratio test}

Choose null hypothesis if,
\begin{equation}
    \frac{p(\textbf{x}|H_0)}{p(\textbf{x}|H_1)} > \gamma
\end{equation}

Choose alternitive hypothesis if,
\begin{equation}
    \frac{p(\textbf{x}|H_0)}{p(\textbf{x}|H_1)} \leq \gamma
\end{equation}

\subsection{Log Likelihood}
Choose null hypothesis if,
\begin{equation}
    -x[0] + \frac{1}{2} > 0 \leftrightarrow x[0] < \frac{1}{2}
\end{equation}

Choose alternitive hypothesis if,
\begin{equation}
    -x[0] + \frac{1}{2} \leq 0 \leftrightarrow x[0] \geq \frac{1}{2}
\end{equation}

Solve for $\gamma$ using,
\begin{equation}
    P_{FA} = \int_{\textbf{x}:L(\textbf{x})<\gamma}p(\textbf{x}|H_0)d\textbf{x}
\end{equation}

\subsection{Missed Detection}

The probability of missed detection is,
\begin{equation}
    P_M = p(x[0] \leq \frac{1}{2}| H_1) = p(x[0] \leq \frac{1}{2}|u = 1)
\end{equation}

\subsection{Detection Error Rate}

\begin{equation}
    P_{FA}P(H_0) + P_MP(H_1)
\end{equation}

Where $P_{FA}$ is the probability of falsely selecting the alternative hypothesis and $P_M$ is the probability
of falsely selecting the null hypothesis.

\subsection{Detection Threshold}


\end{document}
